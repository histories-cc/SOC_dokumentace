\chapter{Zpracování obrázků}
\section{Zpracování při nahrání}
Po obdržení souboru se zkontroluje, jestli se jedná o formát obrázku (např.: .jpg, .png, atd.), pokud je soubor jiného formátu (např.: .zip, .exe, atd.) žádná z následujících akcí nebude provedena a kód je ukončen s errorem. Nejdříve je obrázek zmenšen tak aby jeho nejdelší strana měla maximálně 2560px (poměr stran zůstává zachován) a je převeden do formátu JPG. Následně se fotky asynchronně nahrají na IPFS a u každé je zkontrolováno pomocí externího API, jestli neobsahuje NSFW obsah. Zároveň se pro každou fotku vygeneruje blurhash.

\section{Poskytování obrázků}
Pro zobrazování obrázků je použita komponenta, kterou Next.js obsahuje v základu. Ta umožňuje dynamickou optimalizaci na serveru. V praxi to funguje tak že pokud klient má menší obrazovku a obrázek se vykresluje pouze ve velikosti 300 na 200px tak se neposílá klientovi obrázek v plné velikosti ale zmenšení na konkrétní velikost kterou potřebuje. Všechny fotky jsou caschovány lokálně u klienta a  na Cloudflare pomocí služby [Images.weserv.nl](http://Images.weserv.nl) která umožňuje caschováni zadarmo bez používání API klíče. Před načtením obrázku je na jeho místě vykreslená jeho rozmazaná verze za pomoci blurhashe, protože u blurhashe není potřeba fetchovat žádná dodatečná data.

\section{AVIF}
AVIF je open source formát obrázků. Lze u něj zvolit jestli bude použit při ztrátové, nebo bezztrátové konverzi a oproti formátu JPEG je výstup až o 50\% menší.