\chapter{Kontrola kvality kódu}
\section{Verzování}
Pro verzování je použita technologie Git,
která umožňuje vytvářet verze projektu
a zobrazovat veškerou historii změn.
Zároveň unsadňuje spolupráci více lidí a umožňuje
jednoduché zálohování kódu na platformy jako je
například Gitlab, nebo Github.

\section{Testování}
\subsection{Při vývoji}
\subsubsection{Jest}
Jest je nástroj pro testování jednotlivých funkcí v kódu během vývoje.
Testování probíhá tak, že se zavolá funkce s definovanými parametry,
a testuje se, zda se výsledek shoduje s očekávaným výsledkem.
\subsection{Na produkci}
\subsubsection{Testování produkční verze}
\paragraph{Frontend}
Tyto testy provádí automaticky ovládaný prohlížeč, u kterého je 
definované na jaké místo na obrazovce se má klikat a jaké stránky má navštívit.
Pro testování tohoto typu jsou nejčastěji používány knihovny Selenium a Puppeteer.
Služba která umožňuje toto testování je například Checkly, které i ukládá snímky obrazovky.
\paragraph{API}
API je testováno pomocí speciálního requestu, na který je známa odpověď. Získaná
odpověď se potom musí shodovat s očekávanou. Dále lze testovat například, jak dlouho
trvalo čekání na odpověď.
\subsubsection{Nahlašování vzniklých chyb}
Pokud se při běžném používání aplikace vyskytne error, je nahlášen a 
vývojář se může podívat, v jaké situaci se vyskytl. Tuto funkci má například
služba Sentry.
\section{Analýza kódu}  
\subsection{Prettier}
Prettier je nástroj využívaný v kombinaci s ESLintem. Stará
se o dodržení jednotnosti u neviditelných znaků,
jako jsou například taby, nebo znak nového řádku.
\subsection{Eslint}
Lintery jsou nástroje využívány pro sjednocení syntaxe a 
analýzu kódu ještě před jeho kompilací
a odhalení chyb.
\subsection{Kontrola závislostí}
Závislosti jsou knihovny a moduly, které projekt využívá 
ale musí se doinstalovat zvlášť.
\subsubsection{Kontrola závislostí}
Pro kontrolu zranitelností, které mohou vzniknout v závislostech je 
použit nástroj Snyk, který pravidelně kontroluje bezpečnost závislostí,
v případě nálezu zranitelnosti na ni upozorní, a pokud je nalezena nová
verze závislosti, ve které je tato zranitelnost opravena, navrhne její aktualizaci.
\subsubsection{Aktualizace závislostí}
Pro pravidelnou aktualizaci verzí závislostí je použit nástroj Dependabot spravován Githubem
který pravidelně kontroluje, jestli jsou všechny závislosti co nejaktuálnější
a navrhuje změny v kódu, které závislosti aktualizují.
\subsubsection{CI/CD}
CI je nástroj, nebo sada nástrojů, které automaticky kontrolují kód za použití zmíněných nástrojů.
V tomto projektu je použit nástroj Github actions. Tato kontrola se provádí 
při každém nahrání nové verze kódu. Nejdřív prooběhne kontrola pomocí Prettier a Eslint, pokud je úspěšná,
spustí se pokus o zkompilování projektu. Toto je prováděno pro každou verzi Node.js, se kterou
je projekt kompatibilní (v tomto případě verze 14 a 16). Pokud všechny zmíněné kroky proběhly úspěšně,
je vytvořen nový Docker image s unikátním tagem, pokud se při vytváření nevyskytne chyba, je publikován 
na platformě Dockerhub.

Sem přijde schéma Github actions