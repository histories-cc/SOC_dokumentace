\chapter{Databáze}
\section{Grafové databáze}
\subsection{Co je grafová databáze}
V grafových databázích je použit princip relačního grafování, to znamená, že data jsou znázorněna vrcholy a hranami. Hranou se znázorňueje vztah mezi dvěma (je možno i jedním) vrcholy. Jak vrchol, tak hrana mohou mít vlastnosti, do kterých lze ukládat data.
\subsection{Výhody}
Výhody tento typ databáze přináší v případě, kdy jsou data vzájemně propojená. V paměti je totiž relace uložená jako pointer, na rozdíl od SQL, kde je nutné filtrovat celou tabulku. To umožňuje například vyhledávání propojeních mezi uživateli v reálném čase což by s SQL databází bylo téměř nemožné. Zároveň databáze podporuje ACID a constraints.
\subsection{Dotazovací jazyk}
Dotazovací jazyk je počítačový jazyk, který umožňuje získávání, analýzu a manipulaci s daty v databázi. Neo4j používá jako dotazovací jazyk CQL (Cypher Qeuery Language). Syntaxí je velmi podobný jazyku SQL, který je využíván relačními databázemi. Velmi populární je knihovna APOC, která přidává mnoho užitečných funkcí.
