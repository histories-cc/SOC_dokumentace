\chapter{IPFS}
IPFS je protokol, který umožňuje peer-to-peer sdílení souborů. Každý soubor nahraný na IPFS má unikátní otisk, který se nazývá CID. CID je generován pomocí hash funkce z obsahu souboru, což zamezuje duplikaci. Počítače na kterých jsou data uložena se nazývají nody. Node může být jakýkoliv počítač s připojením k internetu. Jakýkoliv soubor který je na síti může jakýkoliv node připnout, čímž ho stáhne a začne poskytovat dál. To znamená, že každý node poskytuje pouze obsah, ve svém zájmu. Pokud na síti existuje aspoň jeden node, který obsah sdílí, obsah je přístupný odkudkoliv. Výhodou IPFS je, že čím více lidí k souboru přistupuje, tím rychlejší se přístup stává. Nevýhodou je pomalejší načítání dat oproti klasickým objektovým úložištím. Toto lze vyřešit například cachováním.