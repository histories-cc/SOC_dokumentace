% Autor šablony: Jakub Dokulil (kubadokulil99@gmail.com) https://github.com/Kubiczek36/SOC_sablona

\documentclass[12pt, a4paper,
  %oneside,      %% -- odkomentujte, pokud chcete svou práci mít pouze jednostrannou, mezera pro hřbet pak automaticky bude pouze na levé straně
 twoside,        %% -- pro oboustranné práce, mezera pro hřbet následně střídá strany.
 openright
]{report}

%% Nutné balíčky a nastavení
%%%%%%%%%%%%%%%%%%%%%%%%%%%%

\newcommand\city{Pardubice} 
\newcommand\district{Pardubický}
\newcommand\specialization{Obor č. 18: Informatika}
\newcommand\school{SŠIE Delta}
\newcommand\consultant{RNDr. Koupil Jan, Ph.D.}
\newcommand\name{Krátký Kryštof}
\newcommand\publicationYear{2022}

\title{Histories} %% -- Název tvé práce
\author{\name} %% -- tvé jméno
\date{\publicationYear} %% -- rok, kdy píšeš SOČku

\usepackage[top=2.5cm, bottom=2.5cm, left=3.5cm, right=1.5cm]{geometry} %% nastaví okraje, left -- vnitřní okraj, right -- vnější okraj

\usepackage[czech]{babel} %% balík babel pro sazbu v češtině
\usepackage[utf8]{inputenc} %% balíky pro kódování textu m
\usepackage[T1]{fontenc}
\usepackage{cmap} %% balíček zajišťující, že vytvořené PDF bude prohledávatelné a kopírovatelné

\usepackage{graphicx} %% balík pro vkládání obrázků

\usepackage{subcaption} %% balíček pro vkládání podobrázků

\usepackage{hyperref} %% balíček, který v PDF vytváří odkazy

\linespread{1.15} %% řádkování

\usepackage[pagestyles]{titlesec} %% balíček pro úpravu stylu kapitol a sekcí
\titleformat{\chapter}[block]{\scshape\bfseries\LARGE}{\thechapter}{10pt}{\vspace{0pt}}[\vspace{-22pt}]
\titleformat{\section}[block]{\scshape\bfseries\Large}{\thesection}{10pt}{\vspace{0pt}}
\titleformat{\subsection}[block]{\bfseries\large}{\thesubsection}{10pt}{\vspace{0pt}}

\setcounter{secnumdepth}{2}
\setcounter{tocdepth}{1}
\usepackage{fancyhdr}
\pagestyle{fancy}
\renewcommand{\headrulewidth}{1pt}

\usepackage{booktabs}

\usepackage{url}

%% Balíčky co se můžou hodit :) 
%%%%%%%%%%%%%%%%%%%%%%%%%%%%%%%

\usepackage{pdfpages} %% Balíček umožňující vkládat stránky z PDF souborů, 

\usepackage{upgreek} %% Balíček pro sazbu stojatých řeckých písmen, třeba u jednotky mikrometr. Například stojaté mí: \upmu, stojaté pí: \uppi

\usepackage{amsmath}    %% Balíčky amsmath a amsfonts 
\usepackage{amsfonts}   %% pro sazbu matematických symbolů
\usepackage{esint}     %% pro sazbu různých integrálů (např \oiint)
\usepackage{mathrsfs}

%% makra pro sazbu matematiky
\newcommand{\dif}{\mathrm{d}} %% makro pro sazbu diferenciálu, místo toho
%% abych musel psát '\mathrm{d}' mi stačí napsat '\dif' což je mnohem 
%% kratší a mohu si tak usnadnit práci

%% Bordel pro práci - můžeš smáznout :) 
%%%%%%%%%%%%%%%%%%%

\usepackage{lipsum} %% balíček který píše lipsum (nesmyslný text, který se používá pro kontrolu typografie)

%% Začátek dokumentu
%%%%%%%%%%%%%%%%%%%%


\begin{document}

\pagestyle{empty}
\pagenumbering{Roman}


\begin{titlepage}
    \bfseries{ %%% písmo na stránce je tučně
        \begin{center}
            \LARGE{STŘEDOŠKOLSKÁ ODBORNÁ ČINNOST} 
            \vspace{14pt}
            \large{\specialization} 

            \vspace{0.4 \textheight}

            \LARGE{ %%%%
                Histories
            }%%%%

            \vspace{0.4\textheight}
        \end{center}
        
        \noindent\Large{\name}

        \noindent\Large{\district\hspace{\stretch{1}}  \city, \publicationYear} %% vyplň oficiální název kraje, město a rok
        
            
    } %%%
\end{titlepage}

\cleardoublepage%% Úvodní stránka s informacemi
{\bfseries %%% písmo na stránce je tučně
    \begin{center}
        \LARGE{STŘEDOŠKOLSKÁ ODBORNÁ ČINNOST}

        \vspace{14pt}
        {\large \specialization}

        \vspace{0.3 \textheight}

        \LARGE{ %%%%
        Histories
        }

        \LARGE{ %%%%
        platforma pro sdílení historických fotek
        }%%%%

        \vspace{0.24\textheight}
    \end{center}  
}%%%
{\Large %%%
    \noindent\textbf{Jméno:} \name\\
    \textbf{Škola:} \school\\
    \textbf{Kraj:} \district\\
    \textbf{Konzultant:} \consultant\\
} %%%

\noindent \city, \publicationYear\cleardoublepage\noindent{\Large{\bfseries{Prohlášení}}}  %% uprav si koncovky podle toho na jaký rod se cítíš, vypadá to pak lépe :) 

\noindent Prohlašuji, že jsem svou práci SOČ vypracoval/a samostatně a použil/a jsem pouze prameny a literaturu uvedené v seznamu bibliografických záznamů.

\noindent Prohlašuji, že tištěná verze a elektronická verze soutěžní práce SOČ jsou shodné. 

\noindent Nemám závažný důvod proti zpřístupňování této práce v souladu se zákonem č. 121/2000 Sb., o právu autorském, o právech souvisejících s právem autorským a o změně některých zákonů (autorský zákon) ve znění pozdějších předpisů. 

\vspace{24 pt}

\noindent V Pardubicích dne 9.\ září 2022 \dotfill{}\hspace{\stretch{0.5}} 

\hspace{8cm} \name\cleardoublepage\vspace*{0.8\textheight}
\noindent{\Large{\bfseries{Poděkování}}}

\noindent
Rád bych poděkoval svému konzultantovi RNDr. Janu Koupilovi, Ph.D. za odborné vedení a konzultace při tvorbě tohoto projekt.

\cleardoublepage\noindent{\Large{\bfseries{Abstrakt}}}

\noindent TBD abstrakt

\vspace{18pt}

\noindent{\Large{\bfseries{Klíčová slova}}}

\noindent Webová aplikace; sociální síť; sdílení historických fotek

\vspace{18pt}

\noindent{\Large{\bfseries{Abstract}}}

\noindent TBD abstract

\vspace{18pt}

\noindent{\Large{\bfseries{Keywords}}}

\noindent Web app; social media; historical photo sharing

\cleardoublepage\tableofcontents

\pagenumbering{arabic}
\pagestyle{fancy}
\setcounter{page}{1}


\chapter{Pojmy}
\subsection{NSFW}
Zkratka vycházející z anglického "Not Safe For Work" (nevhodné pro práci), zpravidla se takto označují příspěvky obsahující nahotu, násilí a tudíž by se jim mělo vyhnout v profesionálním prostředí.
\subsection{Garbage collection} je správa paměti, Stará se o uvolnění nepoužívaných úseků v paměti které nejsou procesem využívány

\chapter{Úvod}
\section{Cíl}
Cílem projektu bylo vytvořit platformu pro sdílení historických fotografií
s informací o místě a čase pořízení. Jednou z hlavních funkcí aplikace je možnost
zobrazení fotografií na mapě včetně možnosti filtrování podle časové osy. Prohlížet 
příspěvky mají možnost všichni včetně neregistrovaných uživatelů. Registrovaní 
uživatelé dále mohou přidávat nové příspěvky a diskutovat s ostatními uživateli v
komentářové sekci. Zároveň existují administrátorské funkce, které umožňují mazat
nevhodný obsah a přidávat detaily míst (např. název místa, fotku, ikonu na mapě, atd.).

\section{Motivace}
Motivací projektu byla Facebooková skupina Pardubice v běhu času\cite{PardubiceVBehuCasuFB}, kde lidé 
sdílí historické fotky. Na rozdíl od Histories zde není téměř žádná možnost filtrování či 
řazení ani zobrazení příspěvků na mapě.




\section{Existující řešení}
\subsection{Historypin}
Historypin je open-source platforma fungující jako archiv historických fotografií.Umožňuje vyhledávání míst na mapě, filtrování příspěvků podle data vzniku, nebo například vytváření kolekcí. Obashuje velké množství příspěvků, což ale vede k pomalému načítání stránky a dat na mapě. Zvolený layout snižuje přehlednost a zhoršuje orientaci a navigaci na stránce.
\subsection{Facebook}
Primární účel Facebooku je sdílení aktuálních příspěvků. Neobsahuje téměř žádné možnosti filtrování či geolokaci příspěvků. 
\
\chapter{Použité technologie}
\section{Sdílené technologie pro frontend i backend}
\subsection{Typescript}
Typescript je nadstavba na Javascript, která umožňuje statické typování. Všechen kód napsaný v Javascriptu je validním také v Typescriptu. 
\subsection{NPM}
NPM je nástroj pro správu Javascript knihoven. Umožňuje jednoduchou instalaci knihoven použitých v projektu. V konfiguračním souboru lze nastavit nutnost použití konkrétní verze knihovny, což zajišťuje stejnou funkčnost napříč různými zařízeními. 
\subsection{GraphQL}
GraphQL je specifikace dotazovacího jazyka pro tvorbu API.(wikipedia) Nejpoužívanější alternativou pro GraphQL je REST. Oproti REST je u GraphQL výhodou že klient v requestu specifikuje data která chce aby byla vrácena ze serveru. Tímto je eliminován přenos nepotřebných dat a tím také celý přenos urychlen. Zároveň se klient dotazuje vždy na stejný endpoint, mění se jen obsah requestu. GraphQL je typované, což znamená že musí být předem jasné typ vrácených dat.


% FRONTEND %%%%%%% 
\section{Technologie pro frontend}
\subsection{React}
React je Javascriptová knihovna pro tvorbu webových stránek zjednodušující práci s měnícími se daty. Kód je dělen na jednotlivé celky nazývané komponenty. Komponenty mohou být v kódu používány vícekrát, což vede k zjednodušení kódu.
\subsection{Next.js}
Next.js je React framework zaměřený na zlepšení uživatelské i vývojářské zkušenosti. Hlavní výhodou Next.js je server side rendering, což znamená že se data načtou již na serveru a klientovi se posílá načtená stránka, na rozdíl od klasického Reactu, kde se klientovi pošle prázdná stránka a až poté se u klienta načítají data. Server side rendering zároveň zlepšuje skóre SEO. Next.js disponuje vlastními React komponenty, například Link, který zajišťuje prefetchování odkazů nebo Image, který optimalizuje způsob načítání obrázků. Next.js obsahuje také file system rounting, což znamená že není nutné nastavovat React router.
\subsection{Tailwind}
Tailwind je open-source CSS framework, který umožňuje rychlejší a přehlednější psaní stylů stránek pomocí předvytvořených tříd. Tailwind neobsahuje žádné předvytvořené prvky.
\subsection{GraphQL code generator}
GraphQL code generator je knihovna, která umožňuje generování Typescript typů z GraphQL schémata. Díky tomu není nutné vytvářet zvlášť nové typy pro GraphQL response.
% BACKEND %%%%%%% 
\section{tachnologie pro backend}
\subsection{Nodejs}
Node js je runtime prostředí pro spouštění Javascriptu mimo prohlížeč. Hlavním důvodem pro použití bylo použití stejného programovacího jazyka pro backend i frontend, což umožňuje znovuvyužití některých funkcí (např. validace).

\chapter{Úložiště dat}
    \section{Úložiště souborů}
        \subsection{Protokol IPFS}
        Inter Planetary File System (Meziplanetární souborový systém) je peer-to-peer trustless protokol pro ukládání a sdílení souborů v distribuovaném souborovém systému.
        \subsubsection{Decentralizace}
            Decentralizované protokoly nemají vlastníka a jsou utvářeny všemi uživateli společně. Tento koncept často bývá spojen s konceptem trustless, který umožňuje každému uživateli kdykoliv ověřit pravdivost obsahu, v tomto případě vygenerováním hashe a následným porovnáním dvou hashů, čímž je zabráňeno cenzuře. Decentralizace umožňuje rychlejší načtení obsahu, pokud je v blízkosti node s uloženým obsahem, nebo načtení obsahu bez nutnosti přístupu k celosvětové síti, pokud je v lokální síti node, který obsah poskytuje.
        \subsubsection{Zpracování obsahu}
            \paragraph{Nahrávání obshu}
            Při ukládání souboru do IPFS je pomocí algoritmu sha-256 z obsahu souboru vygenerován hash, který slouží jako unikátní identifikátor souboru (CID) pomocí kterého je následně soubor přístupný. Vygenerovaný hash má vždy stejnou délku nezávisle na velikosti daného souboru a při nahrání souboru se shodným obsahem z jakéhokoliv zařízení bude vždy hash totožný, čímž je zajištěna neměnnost obsahu. Nahrané soubory si můžou jednotlivé nody připnout, což je ochrání před garbage collection. Po připnutí je soubor uložen do úložiště nodu a dál poskytován síti. Existují tzv. pinning services, které umožňují pronajmutí nodu, s přístupem k vysokorychlostnímu internetu a je spravován poskytovatelem. V projektu byl použit pinning service Infura. 
            \paragraph{Přístup k obsahu} K obsahu je následně možné přistupovat skrz veřejnou gateway (např. ipfs.io, nebo cf-ipfs.com), nebo skrz gateway nodu, kde jsou připnuté soubory, díky čemuž není nutné vyhledávat nejbližší node poskytující obsah, což celý proces urychluje.
        \subsection{Porovnání IPFS a S3}
            Soubory v IPFS jsou adresovány na základě obsahu na rozdíl od S3, kde jsou sobory adresovány na základě lokace (to umožňuje nahrávat soubory do složek). IPFS neumožňuje nastavení oprávnění k přístupu k souboru, tzn. pokud zná uživatel CID souboru, který je v síti dostupný, nemůže mu být zabráněno v přístupu k obsahu souboru.

    \section{Databáze}
    \subsection{Definice grafové databáze}
        Hlavním rozdílem grafových databází oproti relačním databázím je způsob ukládání dat. 
        Data v grafové databázi jsou rozdělena na uzly a hrany. Uzly představují entity a hrany vztahy znázorňují vztahy mezi entitami. Hrana vždy spojuje dva uzly.
        Uzly i hrany mohou mít své vlastnosti. Grafová databáze byla zvolena, protože aplikace generuje mnoho na sobě závislých dat, která
        lze následně využít u algoritmů pro doporučování příspěvků a ostatních uživatelů v reálném čase. Grafové databáze jsou často využity pro analýzu a zpracování dat.
    \subsection{Neo4j}
        V projektu je použita databáze Neo4j. Databáze byla zvolena zejména kvůli její aktivní komunitě a detailní dokumentaci. Neo4j je používána také některými světovými organizacemi např. NASA, Airbnb, Lyft a Ebay. Neo4j splňuje zásady ACID, což znamená že změny jsou uloženy všechny najednou po dokončení transakce, v případě chyby nejsou žádné změněny uloženy. Tím je garantovaná stálost dat v databázi.
    \subsection{Constraints}
        Constraints je sada omezení při práci s daty. Lze omezovat na základě vlastností vrcholu (např. nelze vytvořit vrchol \uv{user} bez vlastnosti \uv{username}), na základě hran (např. nemůže existovat vrchol \uv{post} bez hrany \uv{created} spojující ho s uživatelem, nemůže existovat hrana sleduje mezi vrcholy \uv{user} a \uv{post}), na základě datových typů (např. vlastnost \uv{email} vrcholu \uv{user} musí být vždy typu string)
    \subsection{Dotazovací jazyk Cypher Query Language}
        Cypher je výchozí dotazovací jazyk pro Neo4j. Syntaxí je podobý jazyku SQL.
    \subsection{Porovnání Neo4j a PostgreSQL}
    \subsection{Práce s daty}
        Podle konvencí Neo4j jsou všechny vrcholy pojmenovány s velkým písmenem na začátku a hrany se všemi písmeny velkými, pokud se jedná o víceslovný název, jednotlivá slova jsou oddělena podtržítkem.
        \subsubsection{Datová struktura}
            \paragraph{Vrcholy}
                \begin{itemize}
    	            \item \textbf{User} (uživatel) - Obsahuje uživatelská data jako jsou např. křestní jméno, příjmení, email, datum a čas vytvoření, hash hesla. Uživatel má společnou hranu \uv{CREATED} s každou entitou, kterou vytvořil (např. \uv{Comment}, \uv{Post}, \uv{Collection}).
    	            \item \textbf{Post} (příspěvek) - Obsahuje datum vytvoření, popisek, jestli se nejedná o nevhodný obsah a historický datum ve formátu popsaném v sekci \ref{section:historica_date}. Vrchol \uv{Photo} může mít společnou hranu \uv{BELONGS\_TO} s vrcholem \uv{Post}, kterým je fotografie nahraná k příspěvku. Příspěvek má společnou hranu \uv{IS\_LOCATED} s vrcholem \uv{Place}, tímto se zajistí přiřazovaní příspěvků k místům.
    	            \item \textbf{Photo} (fotografie) - Každá použitá fotografie je uložena jako vrchol. Obsahuje vlastnosti blurhash (\ref{TBD}), hash (IPFS content id \ref{}), výšku a šířku (pro správné vykreslování blurhashe na frontendu), index (v případě že je k jednomu příspěvku přiřazeno více fotografií, řadí se podle této vlastnosti).
    	            \item \textbf{Place} (místo) - Znázorňuje jedno místo na mapě. Obsahuje vlastnosti location (je použit datový typ point, pro jednodušší zpracování), název a popisek. Může mít hranu \uv{HAS\_PREVIEW} s vrcholem \uv{Photo}, vrchol \uv{Photo} je potom náhled místa, zobrazující se na mapě, nebo v jeho detailu.
    	            \item \textbf{Comment} (komentář) - Má společnou hranu \uv{BELONGS\_TO} s vrcholem \uv{Post}. Obsahuje vlastnost \uv{content} (text komentáře). Může mít společnou hranu \uv{LIKE} s vrcholem \uv{User}, pokud ho nějaký uživatel označí jako oblíbený.
    	            \item \textbf{Collection} (kolekce) - Má společnou hranu \uv{CONTAINS} s vrcholy \uv{Post}. Obsahuje vlastnosti název a popis.
            	\end{itemize}
            \paragraph{Hrany}
                \begin{itemize}
    	            \item \textbf{FOLLOW} (sleduje) - Může být pouze mezi vrcholy \uv{User}.
    	            \item \textbf{LIKE} (líbí se) - Může být pouze mezi vrcholem \uv{User} a \uv{Post}.
    	            \item \textbf{REPORT} (nahlášení) - Může být pouze mezi vrcholem \uv{User} a \uv{Post}.
                \end{itemize}
        \subsubsection{Doporučování uživatelů}

\chapter{Zabezpečení}
\section{Validace vstupů}
\subsection{GraphQL}
GraphQL vyžaduje, aby uživatelův vstup byl stejného typu jako očekává (uživatel na místo čísla nemůže zadat text), tato validace je provedena ještě před začátkem akce na backendu. 
\subsection{Validace textových vstupů}
\paragraph{Uživatelské jméno} - musí být unikátní, může obsahovat malá a velká písmena bez diakritiky, čísla a podtržítko.
\paragraph{Popisky a komentáře} - mohou obsahovat jakékoliv znaky, jsou omezeny poze maximální délkou. Všechny speciální znaky jsou escapovány pro zabránění injectu do databáze.
\subsection{Souřadnice} Při vatváření příspěvku je jeho umístění validováno podle standardu ISO 6709.
\subsection{Soubory} Middleware omezuje maximální počet najednou nahrávaných souborů na 5 při maximální velikosti jednoho souboru 20MB.
\subsection{Historické datum}
    \begin{itemize}
        \item \textbf{Validace přesného data} - pokud je uživatelem zadáno přesné datum, je pro ověření jeho správnosti použita funkce JavaScriptu pro rozpoznávání dat.
        \item \textbf{Validace data s chybějícím dnem nebo měsícem} - pokud je vyplněn pouze den a rok, den bude nastaven jako null z logických důvodů. 
        \item \textbf{Validace roku} - minimální povolený hodnota je 1400, v případě potřeby ji lze změnit v konfiguračním souboru. Maximální hodnota je vždy aktuální rok.        
        \item \textbf{Validace měsíce} - měsíc je zadán jako číslo od 1 do 12.
        \item \textbf{Časové rozmezí} - pokud datum je konce rozmezí starší než datum začátku jsou tyto dvě hodnoty prohozeny. Pokud je datum začátku a konce shodné, jedná se o přesné datum (nejedná se o rozmezí), tento formát je validní.
    \end{itemize}


\chapter{Praktická část}


\chapter{Obrázky}
\section{Zpracování při nahrání}
\subsection{Filtrování nevhodného obsahu}\label{section:nsfw_filter}
Pro kontrolu NSFW obsahu autor použil API třetí strany (nsfw iamge classification1 Rapid API). Toto API bylo zvoleno z důvodu vysoké kvality vytrénovaného modelu a tudíž velké úspěšnosti rozpoznání nevhodného obsahu a malé míry falešně pozitivních výsledků (false positive). Příspěvky obsahující fotografii klasifikovanou jako NSFW jsou vytvořeny, ale jejich viditelnost je nastavena na skrytou, dokud některý z adminů tento příspěvek buď zamítne (poté je smazán), nebo potvrdí (je nastaven jako viditelný, s vlastností NSFW). Pokud se tento příspěvek poté má zobrazit uživateli, nejdřív je zobrazena rozmazaná verze s varováním, po jehož přijmutí lze zobrazit fotografii.

\subsection{Validace souboru} Maximální velikost nahraného obrázku je 20MB.  Po nahrání je na backendu ověřeno, zda se jedná o obrázek (např. PNG, JPG, GIF), poté je převeden do formátu JPG. Pokud nejdelší strana obrázku přesahuje 2560px, obrázek je zmenšen při zachování poměru stran tak, aby jeho nejdelší strana měla 2560px. 
\section{Optimalizace}
\subsection{Použití komponentu Next image}
Next.js poskytuje React komponentu Image, která umožňuje využívat optimalizaci obrázků (tzn. rychlejší načtení na straně klienta). V argumentech komponenty lze nastavit kvalitu obrázku a tím zmenšit datový objem nutný doručit klientovi. Zároveň obrázek není přenášen v plném rozlišení, ale je přizpůsobeno velikosti, ve které je obrázek vykreslen na obrazovce. 
\subsection{Porovnání AVIF a WEBp}
\section{Blurhash}
Blurhash je formát vyvinut a využíván společností Wolt. Umožňuje vygenerování krátkého řetězce znaků z fotogrfie. Řetězec znaků lze následně vykreslit jako rozmazanou verzi fotografie, která je následně vykreslena na místě fotografie před jejím načtením. Tento detail zajišťuje lepší uživatelskou zkušenost a pocit rychlejšího načítání fotografií.



\section{Historický datum příspěvku}\label{section:historica_date}
Každý příspěvek má datum kdy byl vytvořen a historické datum. Historické datum je datum kdy byla fotografie u příspěvku pořízena (z pravidla starší data). U takto starých dat je nutné počítat s tím že se může jedna pouze o odhad, proto není možné použít datum ve formátu podle standartu (ISO 8601) jak je zvykem. V autorem navržené struktuře se počítá s tím, že je znám alespoň rok, jako dodatečné informace lze přidat i měsíc, případně den. Zároveň se ale může jednat o určité časové období (například červen až červenec roku 1984), proto jsou vlastnosti rozděleny na počáteční a konečné datum. V případě že se o období nejedná, jsou počáteční a konečné datum stejné. V praxi jde o šest vlastnosí, které má každý příspěvek: startDay, startMonth, startYear, endDay, endMonth, endYear.

\section{Zobrazení mapy}
    Pro zobrazení mapy je použit balíček react-map-gl, který umožňuje zobrazování např. bodů, polygonů a 3D terénu. Pro zobrazení konkrétních stylů mapy je použit tiling service. Mapa je rozdělena do mřížky ve které jedno pole je jeden tzv. tile, pro tile je poté ze serveru poslán obrázek na kterém je konkrétní část mapy (existují i varianty které používají místo obrázků vektorovou grafiku, což zvyšuje rychlost načítání, a eliminuje efekt rozmatanosti), výhodou je že se ze serveru posílá několik tilů zároveň a ty se poté cachují u klienta. Díky tomu není nutné stahovat data pro jednu část mapy vícekrát. V projektu byl jako tiling service použit Mapbox, jedná se o placené řešení s možností použití předvytvořených stylů mapy, nebo vytvoření zcela vlastního. Alternativou je např. Leaflet, který je poháněn open-source projektem open street maps, který ale obsahuje pouze několik stylů bez možnosti vlastní úpravy.
    \subsection{Geolokace klienta}
    \subsection{Ukládání poslední polohy na mapě}
    \subsection{Dotazy na server}
    \subsection{Seskupování bodů}
    Místa na mapě jsou zobrazena jako ikony s fotografií místa. Pokud by bylo mnoho bodů na malém prostoru, mohly by se nevzájem překrývat, snižovat přehlednost a zpomalovat pohyb po mapě. Proto je použit tzv. clustering (shlukování více bodů), místo těchto bodů je pak vykreslen pouze jeden, který se nachází na souřadnicích rovnajících se průměru souřadnic všech bodů, které cluster obsahuje. Pro rozhodnutí, který obrázek bude na mapě použit jako ikona jsou porovnány vzdálenosti všech bodů v clusteru od středu clusteru, a jako náhled je použit náhled místa, které je nejblíže středu clusteru. V pravém vrchním rohu ikony je poté číslo vyjadřující počet míst seskupených v clusteru.


\chapter{Docker}
Docker je platforma která umožňuje vytvořit virtuální prostředí pro 
běh aplikace, které je odděleno od hostitelského prostředí.
Výhodou jsou předem dané podmínky pro běh aplikace, které zůstávají
bez ohledu na počítač na kterém toto prostředí beží. 
\subsection{Docker image}
Docker image je soubor, který je použit pro spuštění kontejneru, který
může obsahovat například aplikaci. Výhodou je spuštění aplikace,
nebo celého vývojového prostředí (databáze, úložiště soubory, atd.) pomcí
jednoho příkazu.

\chapter{Kontrola kvality kódu}
\section{Verzování}
GIT
\section{Testování}
\subsection{Při vývoji}  

\subsection{Na produkci} 
\subsubsection{Vizuální testy}
\subsubsection{API}
\subsubsection{Nahlašování vzniklých chyb}


 
\section{Formátotvání kódu}  
\subsection{Prettier}
\subsection{Eslint}


\section{CI/CD}
\subsection{Aktualizace knihoven}
\subsection{Kontrola zranitelností}
\subsection{Github actions} 


%% Citace
%% citace
\begin{thebibliography}{55}
    \bibitem{PardubiceVBehuCasuFB} 
    Zajímavosti, videa a obrázky z historie Pardubic \textit{Pardubice v běhu času} [online]. [cit. 2022-03-19]. Dostupné z: \url{https://www.facebook.com/groups/starepardubice/}

    \bibitem{Neo4jSparialFunctions} 
    Spatial functions - Neo4j Cypher Manual \textit{Neo4j} [online]. [cit. 2022-03-19]. Dostupné z: \url{https://neo4j.com/docs/cypher-manual/4.4/functions/spatial/}
    
    \bibitem{Neo4jNamingRules} 
    Naming rules and recommendations - Neo4j Cypher Manual \textit{Neo4j} [online]. [cit. 2022-03-19]. Dostupné z: \url{https://neo4j.com/docs/cypher-manual/4.4/syntax/naming/}
    
    \bibitem{IPFSDocs} 
    IPFS is a distributed system for storing and accessing files, websites, applications, and data \textit{IPFS} [online]. [cit. 2022-03-19]. Dostupné z: \url{https://docs.ipfs.io/}
    
    \bibitem{GarbageCollection} 
    The Very Basics of Garbage Collection \textit{basen\.oru\.se} [online]. [cit. 2022-03-19]. Dostupné z: \url{http://basen.oru.se/kurser/koi/2008-2009-p1/texter/gc/index.html/}
    
    \bibitem{IPFSPersistence} 
    Persistence | IPFS Docs \textit{IPFS} [online]. [cit. 2022-03-19]. Dostupné z: \url{https://docs.ipfs.io/concepts/persistence/}

    \bibitem{NSFW} 
    NSFW Meaning, what does NSFW mean in text \textit{myenglishteacher.eu} [online]. [cit. 2022-03-19]. Dostupné z: \url{https://www.myenglishteacher.eu/blog/nsfw-meaning/}
    
    \bibitem{whatIsTypescript} 
    What is typescript \textit{typescripttutorial.net} [online]. [cit. 2022-03-19]. Dostupné z: \url{https://www.typescripttutorial.net/typescript-tutorial/what-is-typescript/}
    
    \bibitem{typescriptForTheNewProgrammer} 
    TypeScript for the New Programmer \textit{Typescript docs} [online]. [cit. 2022-03-19]. Dostupné z: \url{https://www.typescriptlang.org/docs/handbook/typescript-from-scratch.html/}
    
    \bibitem{DynamicVsStaticTyping} 
    Dynamic typing vs. static typing \textit{Oracle docs} [online]. [cit. 2022-03-19]. Dostupné z: \url{https://docs.oracle.com/cd/E57471_01/bigData.100/extensions_bdd/src/cext_transform_typing.html}
    
    \bibitem{whatIsNpmW3} 
    What is npm \textit{W3 schools} [online]. [cit. 2022-03-19]. Dostupné z: \url{https://www.w3schools.com/whatis/whatis_npm.asp}

    \bibitem{aboutNpm} 
    About npm \textit{NPM docs} [online]. [cit. 2022-03-19]. Dostupné z: \url{https://docs.npmjs.com/about-npm}
    
    \bibitem{graphqlIntroduction} 
    Introduction to GraphQL \textit{GraphQL} [online]. [cit. 2022-03-19]. Dostupné z: \url{https://graphql.org/learn/}

    \bibitem{restApiRedHat}
    What is REST API? \textit{Red Hat} [online]. [cit. 2022-03-19]. Dostupné z: \url{https://www.redhat.com/en/topics/api/what-is-a-rest-api/}
    
    \bibitem{whatIsApiAws}
    What is an API? \textit{Amazon AWS} [online]. [cit. 2022-03-19]. Dostupné z: \url{https://aws.amazon.com/what-is/api/}
    
    \bibitem{graphqlTypes}
    Schemas and Types \textit{GraphQL} [online]. [cit. 2022-03-19]. Dostupné z: \url{https://graphql.org/learn/schema/#type-system}
    
        
    \bibitem{gettingStartedReact}
    Getting started \textit{React docs} [online]. [cit. 2022-03-19]. Dostupné z: \url{https://reactjs.org/docs/getting-started.html}
            
    \bibitem{reactComponentsAndProps}
    Components and Props \textit{React docs} [online]. [cit. 2022-03-19]. Dostupné z: \url{https://reactjs.org/docs/components-and-props.html}
    
    \bibitem{nextImage}
    API documentation for the Image Component and Image Optimization \textit{Next.js docs} [online]. [cit. 2022-03-19]. Dostupné z: \url{https://nextjs.org/docs/api-reference/next/image}

    \bibitem{nextGetStarted}
    Gettings started \textit{Next.js docs} [online]. [cit. 2022-03-19]. Dostupné z: \url{https://nextjs.org/docs/getting-started}
    
    \bibitem{whatIsSEO}
    What Is SEO \textit{Next.js docs} [online]. [cit. 2022-03-19]. Dostupné z: \url{https://moz.com/learn/seo/what-is-seo}
    
    
    \bibitem{whatIsSSR}
    Client-side vs. server-side rendering \textit{Free code camp} [online]. [cit. 2022-03-19]. Dostupné z: \url{https://www.freecodecamp.org/news/what-exactly-is-client-side-rendering-and-hows-it-different-from-server-side-rendering-bd5c786b340d/}
    
    
    \bibitem{blurhash}
    BlurHash is a compact representation of a placeholder for an image \textit{Blurhash} [online]. [cit. 2022-03-19]. Dostupné z: \url{https://blurha.sh/}
     
    \bibitem{blurhashWoltBlog}
    How we came to create a new image placeholder algorithm, BlurHash \textit{Wolt blog} [online]. [cit. 2022-03-19]. Dostupné z: \url{https://blog.wolt.com/hq/2019/07/01/how-we-came-to-create-a-new-image-placeholder-algorithm-blurhash/}
    
    \bibitem{tailwindDocumentation}
    Get started with Tailwind CSS \textit{Tailwind CSS} [online]. [cit. 2022-03-19]. Dostupné z: \url{https://tailwindcss.com/docs/installation/}
        
    \bibitem{graphqlCodeGeneratorDocs}
    Introduction to GraphQL Code Generator \textit{GraphQL code generator docs} [online]. [cit. 2022-03-19]. Dostupné z: \url{https://www.graphql-code-generator.com/docs/getting-started}
            
    \bibitem{whatIsRuntimeEnvironment}
    What Does Runtime Environment (RTE) Mean \textit{Techopedia} [online]. [cit. 2022-03-19]. Dostupné z: \url{https://www.techopedia.com/definition/5466/runtime-environment-rte}
    
                
    \bibitem{aboutNodeJS}
    About Node.js® \textit{Node.js®} [online]. [cit. 2022-03-19]. Dostupné z: \url{https://nodejs.org/en/about/}
    
                    
    \bibitem{decentralizedVsTrustless}
    Decentralized vs Trustless Networks \textit{Bluzelle} [online]. [cit. 2022-03-19]. Dostupné z: \url{https://bluzelle.com/blog/decentralized-vs-trustless-networks}

    \bibitem{decentralizedAndTrustless}
    Decentralized and Trustless Networks \textit{Hackernoon} [online]. [cit. 2022-03-19]. Dostupné z: \url{https://hackernoon.com/decentralized-and-trustless-networks-f881671fae4e}
    
    \bibitem{decentralizedAndDistributedNetworks}
    Centralized, Decentralized, \& Distributed Networks \textit{Cryptopedia} [online]. [cit. 2022-03-19]. Dostupné z: \url{https://www.gemini.com/cryptopedia/blockchain-network-decentralized-distributed-centralized}
    
        
    \bibitem{IPFScontentAddressing}
    Content addressing and CIDs \textit{IPFS} [online]. [cit. 2022-03-19]. Dostupné z: \url{https://docs.ipfs.io/concepts/content-addressing/}
             
    \bibitem{IPFSgateways}
    IPFS Gateway \textit{IPFS} [online]. [cit. 2022-03-19]. Dostupné z: \url{https://docs.ipfs.io/concepts/ipfs-gateway/}
    
    \bibitem{graphDatabasesIntroduction}
    Graph Databases for Beginners: Why Graph Technology Is the Future \textit{Neo4j} [online]. [cit. 2022-03-19]. Dostupné z: \url{https://neo4j.com/blog/why-graph-databases-are-the-future/}
     
    \bibitem{graphenTheorie}
    Graphentheorie \textit{Mathepedia} [online]. [cit. 2022-03-19]. Dostupné z: \url{https://mathepedia.de/Graphentheorie.html}
     
    
    \bibitem{graphAlgorithms}
    Graph Algorithms in Neo4j: Neo4j Graph Analytics \textit{Neo4j} [online]. [cit. 2022-03-19]. Dostupné z: \url{https://neo4j.com/blog/graph-algorithms-in-neo4j-neo4j-graph-analytics/}
     
        
    \bibitem{aboutNeo4j}
    The Fastest Path to Graph \textit{Neo4j} [online]. [cit. 2022-03-19]. Dostupné z: \url{https://neo4j.com}
            
    \bibitem{ACIDRules}
    Database Talk: What is ACID compliance? \textit{FairCom} [online]. [cit. 2022-03-19]. Dostupné z: \url{https://www.faircom.com/insights/database-talk-acid-compliance}
     
                
    \bibitem{CypherQL}
    Cypher Query Language \textit{Neo4j} [online]. [cit. 2022-03-19]. Dostupné z: \url{https://neo4j.com/developer/cypher/}
     
    
    
    
\end{thebibliography}


%% Seznam literatury
\begin{thebibliography}{9}
    \bibitem{texbook}
    Neo4j documentation, online: https://neo4j.com/docs/
    
    \end{thebibliography} 

\end{document}