% Autor šablony: Jakub Dokulil (kubadokulil99@gmail.com) https://github.com/Kubiczek36/SOC_sablona

\documentclass[12pt, a4paper,
  %oneside,      %% -- odkomentujte, pokud chcete svou práci mít pouze jednostrannou, mezera pro hřbet pak automaticky bude pouze na levé straně
 twoside,        %% -- pro oboustranné práce, mezera pro hřbet následně střídá strany.
 openright
]{report}

%% Nutné balíčky a nastavení
%%%%%%%%%%%%%%%%%%%%%%%%%%%%

\newcommand\city{Pardubice} 
\newcommand\district{Pardubický}
\newcommand\specialization{Obor č. 18: Informatika}
\newcommand\school{SŠIE Delta}
\newcommand\consultant{RNDr. Koupil Jan, Ph.D.}
\newcommand\name{Krátký Kryštof}
\newcommand\publicationYear{2022}

\title{Histories} %% -- Název tvé práce
\author{\name} %% -- tvé jméno
\date{\publicationYear} %% -- rok, kdy píšeš SOČku

\usepackage[top=2.5cm, bottom=2.5cm, left=3.5cm, right=1.5cm]{geometry} %% nastaví okraje, left -- vnitřní okraj, right -- vnější okraj

\usepackage[czech]{babel} %% balík babel pro sazbu v češtině
\usepackage[utf8]{inputenc} %% balíky pro kódování textu m
\usepackage[T1]{fontenc}
\usepackage{cmap} %% balíček zajišťující, že vytvořené PDF bude prohledávatelné a kopírovatelné

\usepackage{graphicx} %% balík pro vkládání obrázků

\usepackage{subcaption} %% balíček pro vkládání podobrázků

\usepackage{hyperref} %% balíček, který v PDF vytváří odkazy

\linespread{1.15} %% řádkování

\usepackage[pagestyles]{titlesec} %% balíček pro úpravu stylu kapitol a sekcí
\titleformat{\chapter}[block]{\scshape\bfseries\LARGE}{\thechapter}{10pt}{\vspace{0pt}}[\vspace{-22pt}]
\titleformat{\section}[block]{\scshape\bfseries\Large}{\thesection}{10pt}{\vspace{0pt}}
\titleformat{\subsection}[block]{\bfseries\large}{\thesubsection}{10pt}{\vspace{0pt}}

\setcounter{secnumdepth}{2}
\setcounter{tocdepth}{1}
\usepackage{fancyhdr}
\pagestyle{fancy}
\renewcommand{\headrulewidth}{1pt}

\usepackage{booktabs}

\usepackage{url}

%% Balíčky co se můžou hodit :) 
%%%%%%%%%%%%%%%%%%%%%%%%%%%%%%%

\usepackage{pdfpages} %% Balíček umožňující vkládat stránky z PDF souborů, 

\usepackage{upgreek} %% Balíček pro sazbu stojatých řeckých písmen, třeba u jednotky mikrometr. Například stojaté mí: \upmu, stojaté pí: \uppi

\usepackage{amsmath}    %% Balíčky amsmath a amsfonts 
\usepackage{amsfonts}   %% pro sazbu matematických symbolů
\usepackage{esint}     %% pro sazbu různých integrálů (např \oiint)
\usepackage{mathrsfs}

%% makra pro sazbu matematiky
\newcommand{\dif}{\mathrm{d}} %% makro pro sazbu diferenciálu, místo toho
%% abych musel psát '\mathrm{d}' mi stačí napsat '\dif' což je mnohem 
%% kratší a mohu si tak usnadnit práci

%% Bordel pro práci - můžeš smáznout :) 
%%%%%%%%%%%%%%%%%%%

\usepackage{lipsum} %% balíček který píše lipsum (nesmyslný text, který se používá pro kontrolu typografie)

%% Začátek dokumentu
%%%%%%%%%%%%%%%%%%%%


\begin{document}

\pagestyle{empty}
\pagenumbering{Roman}


\begin{titlepage}
    \bfseries{ %%% písmo na stránce je tučně
        \begin{center}
            \LARGE{STŘEDOŠKOLSKÁ ODBORNÁ ČINNOST} 
            \vspace{14pt}
            \large{\specialization} 

            \vspace{0.4 \textheight}

            \LARGE{ %%%%
                Histories
            }%%%%

            \vspace{0.4\textheight}
        \end{center}
        
        \noindent\Large{\name}

        \noindent\Large{\district\hspace{\stretch{1}}  \city, \publicationYear} %% vyplň oficiální název kraje, město a rok
        
            
    } %%%
\end{titlepage}

\cleardoublepage%% Úvodní stránka s informacemi
{\bfseries %%% písmo na stránce je tučně
    \begin{center}
        \LARGE{STŘEDOŠKOLSKÁ ODBORNÁ ČINNOST}

        \vspace{14pt}
        {\large \specialization}

        \vspace{0.3 \textheight}

        \LARGE{ %%%%
        Histories
        }

        \LARGE{ %%%%
        platforma pro sdílení historických fotek
        }%%%%

        \vspace{0.24\textheight}
    \end{center}  
}%%%
{\Large %%%
    \noindent\textbf{Jméno:} \name\\
    \textbf{Škola:} \school\\
    \textbf{Kraj:} \district\\
    \textbf{Konzultant:} \consultant\\
} %%%

\noindent \city, \publicationYear\cleardoublepage\noindent{\Large{\bfseries{Prohlášení}}}  %% uprav si koncovky podle toho na jaký rod se cítíš, vypadá to pak lépe :) 

\noindent Prohlašuji, že jsem svou práci SOČ vypracoval/a samostatně a použil/a jsem pouze prameny a literaturu uvedené v seznamu bibliografických záznamů.

\noindent Prohlašuji, že tištěná verze a elektronická verze soutěžní práce SOČ jsou shodné. 

\noindent Nemám závažný důvod proti zpřístupňování této práce v souladu se zákonem č. 121/2000 Sb., o právu autorském, o právech souvisejících s právem autorským a o změně některých zákonů (autorský zákon) ve znění pozdějších předpisů. 

\vspace{24 pt}

\noindent V Pardubicích dne 9.\ září 2022 \dotfill{}\hspace{\stretch{0.5}} 

\hspace{8cm} \name\cleardoublepage\vspace*{0.8\textheight}
\noindent{\Large{\bfseries{Poděkování}}}

\noindent
Rád bych poděkoval svému konzultantovi RNDr. Janu Koupilovi, Ph.D. za odborné vedení a konzultace při tvorbě tohoto projekt.

\cleardoublepage\noindent{\Large{\bfseries{Abstrakt}}}

\noindent TBD abstrakt

\vspace{18pt}

\noindent{\Large{\bfseries{Klíčová slova}}}

\noindent Webová aplikace; sociální síť; sdílení historických fotek

\vspace{18pt}

\noindent{\Large{\bfseries{Abstract}}}

\noindent TBD abstract

\vspace{18pt}

\noindent{\Large{\bfseries{Keywords}}}

\noindent Web app; social media; historical photo sharing

\cleardoublepage\tableofcontents

\pagenumbering{arabic}
\pagestyle{fancy}
\setcounter{page}{1}

\chapter{Slovník}


\chapter{Úvod}
\section{Cíl}
Cílem projektu bylo vytvořit platformu pro sdílení historických fotografií
s informací o místě a čase pořízení. Jednou z hlavních funkcí aplikace je možnost
zobrazení fotografií na mapě včetně možnosti filtrování podle časové osy. Prohlížet 
příspěvky mají možnost všichni včetně neregistrovaných uživatelů. Registrovaní 
uživatelé dále mohou přidávat nové příspěvky a diskutovat s ostatními uživateli v
komentářové sekci. Zároveň existují administrátorské funkce, které umožňují mazat
nevhodný obsah a přidávat detaily míst (např. název místa, fotku, ikonu na mapě, atd.).

\section{Motivace}
Motivací projektu byla Facebooková skupina Pardubice v běhu času\cite{PardubiceVBehuCasuFB}, kde lidé 
sdílí historické fotky. Na rozdíl od Histories zde není téměř žádná možnost filtrování či 
řazení ani zobrazení příspěvků na mapě.



%% Citace
%% citace
\begin{thebibliography}{55}
    \bibitem{PardubiceVBehuCasuFB} 
    Zajímavosti, videa a obrázky z historie Pardubic \textit{Pardubice v běhu času} [online]. [cit. 2022-03-19]. Dostupné z: \url{https://www.facebook.com/groups/starepardubice/}

    \bibitem{Neo4jSparialFunctions} 
    Spatial functions - Neo4j Cypher Manual \textit{Neo4j} [online]. [cit. 2022-03-19]. Dostupné z: \url{https://neo4j.com/docs/cypher-manual/4.4/functions/spatial/}
    
    \bibitem{Neo4jNamingRules} 
    Naming rules and recommendations - Neo4j Cypher Manual \textit{Neo4j} [online]. [cit. 2022-03-19]. Dostupné z: \url{https://neo4j.com/docs/cypher-manual/4.4/syntax/naming/}
    
    \bibitem{IPFSDocs} 
    IPFS is a distributed system for storing and accessing files, websites, applications, and data \textit{IPFS} [online]. [cit. 2022-03-19]. Dostupné z: \url{https://docs.ipfs.io/}
    
    \bibitem{GarbageCollection} 
    The Very Basics of Garbage Collection \textit{basen\.oru\.se} [online]. [cit. 2022-03-19]. Dostupné z: \url{http://basen.oru.se/kurser/koi/2008-2009-p1/texter/gc/index.html/}
    
    \bibitem{IPFSPersistence} 
    Persistence | IPFS Docs \textit{IPFS} [online]. [cit. 2022-03-19]. Dostupné z: \url{https://docs.ipfs.io/concepts/persistence/}

    \bibitem{NSFW} 
    NSFW Meaning, what does NSFW mean in text \textit{myenglishteacher.eu} [online]. [cit. 2022-03-19]. Dostupné z: \url{https://www.myenglishteacher.eu/blog/nsfw-meaning/}
    
    \bibitem{whatIsTypescript} 
    What is typescript \textit{typescripttutorial.net} [online]. [cit. 2022-03-19]. Dostupné z: \url{https://www.typescripttutorial.net/typescript-tutorial/what-is-typescript/}
    
    \bibitem{typescriptForTheNewProgrammer} 
    TypeScript for the New Programmer \textit{Typescript docs} [online]. [cit. 2022-03-19]. Dostupné z: \url{https://www.typescriptlang.org/docs/handbook/typescript-from-scratch.html/}
    
    \bibitem{DynamicVsStaticTyping} 
    Dynamic typing vs. static typing \textit{Oracle docs} [online]. [cit. 2022-03-19]. Dostupné z: \url{https://docs.oracle.com/cd/E57471_01/bigData.100/extensions_bdd/src/cext_transform_typing.html}
    
    \bibitem{whatIsNpmW3} 
    What is npm \textit{W3 schools} [online]. [cit. 2022-03-19]. Dostupné z: \url{https://www.w3schools.com/whatis/whatis_npm.asp}

    \bibitem{aboutNpm} 
    About npm \textit{NPM docs} [online]. [cit. 2022-03-19]. Dostupné z: \url{https://docs.npmjs.com/about-npm}
    
    \bibitem{graphqlIntroduction} 
    Introduction to GraphQL \textit{GraphQL} [online]. [cit. 2022-03-19]. Dostupné z: \url{https://graphql.org/learn/}

    \bibitem{restApiRedHat}
    What is REST API? \textit{Red Hat} [online]. [cit. 2022-03-19]. Dostupné z: \url{https://www.redhat.com/en/topics/api/what-is-a-rest-api/}
    
    \bibitem{whatIsApiAws}
    What is an API? \textit{Amazon AWS} [online]. [cit. 2022-03-19]. Dostupné z: \url{https://aws.amazon.com/what-is/api/}
    
    \bibitem{graphqlTypes}
    Schemas and Types \textit{GraphQL} [online]. [cit. 2022-03-19]. Dostupné z: \url{https://graphql.org/learn/schema/#type-system}
    
        
    \bibitem{gettingStartedReact}
    Getting started \textit{React docs} [online]. [cit. 2022-03-19]. Dostupné z: \url{https://reactjs.org/docs/getting-started.html}
            
    \bibitem{reactComponentsAndProps}
    Components and Props \textit{React docs} [online]. [cit. 2022-03-19]. Dostupné z: \url{https://reactjs.org/docs/components-and-props.html}
    
    \bibitem{nextImage}
    API documentation for the Image Component and Image Optimization \textit{Next.js docs} [online]. [cit. 2022-03-19]. Dostupné z: \url{https://nextjs.org/docs/api-reference/next/image}

    \bibitem{nextGetStarted}
    Gettings started \textit{Next.js docs} [online]. [cit. 2022-03-19]. Dostupné z: \url{https://nextjs.org/docs/getting-started}
    
    \bibitem{whatIsSEO}
    What Is SEO \textit{Next.js docs} [online]. [cit. 2022-03-19]. Dostupné z: \url{https://moz.com/learn/seo/what-is-seo}
    
    
    \bibitem{whatIsSSR}
    Client-side vs. server-side rendering \textit{Free code camp} [online]. [cit. 2022-03-19]. Dostupné z: \url{https://www.freecodecamp.org/news/what-exactly-is-client-side-rendering-and-hows-it-different-from-server-side-rendering-bd5c786b340d/}
    
    
    \bibitem{blurhash}
    BlurHash is a compact representation of a placeholder for an image \textit{Blurhash} [online]. [cit. 2022-03-19]. Dostupné z: \url{https://blurha.sh/}
     
    \bibitem{blurhashWoltBlog}
    How we came to create a new image placeholder algorithm, BlurHash \textit{Wolt blog} [online]. [cit. 2022-03-19]. Dostupné z: \url{https://blog.wolt.com/hq/2019/07/01/how-we-came-to-create-a-new-image-placeholder-algorithm-blurhash/}
    
    \bibitem{tailwindDocumentation}
    Get started with Tailwind CSS \textit{Tailwind CSS} [online]. [cit. 2022-03-19]. Dostupné z: \url{https://tailwindcss.com/docs/installation/}
        
    \bibitem{graphqlCodeGeneratorDocs}
    Introduction to GraphQL Code Generator \textit{GraphQL code generator docs} [online]. [cit. 2022-03-19]. Dostupné z: \url{https://www.graphql-code-generator.com/docs/getting-started}
            
    \bibitem{whatIsRuntimeEnvironment}
    What Does Runtime Environment (RTE) Mean \textit{Techopedia} [online]. [cit. 2022-03-19]. Dostupné z: \url{https://www.techopedia.com/definition/5466/runtime-environment-rte}
    
                
    \bibitem{aboutNodeJS}
    About Node.js® \textit{Node.js®} [online]. [cit. 2022-03-19]. Dostupné z: \url{https://nodejs.org/en/about/}
    
                    
    \bibitem{decentralizedVsTrustless}
    Decentralized vs Trustless Networks \textit{Bluzelle} [online]. [cit. 2022-03-19]. Dostupné z: \url{https://bluzelle.com/blog/decentralized-vs-trustless-networks}

    \bibitem{decentralizedAndTrustless}
    Decentralized and Trustless Networks \textit{Hackernoon} [online]. [cit. 2022-03-19]. Dostupné z: \url{https://hackernoon.com/decentralized-and-trustless-networks-f881671fae4e}
    
    \bibitem{decentralizedAndDistributedNetworks}
    Centralized, Decentralized, \& Distributed Networks \textit{Cryptopedia} [online]. [cit. 2022-03-19]. Dostupné z: \url{https://www.gemini.com/cryptopedia/blockchain-network-decentralized-distributed-centralized}
    
        
    \bibitem{IPFScontentAddressing}
    Content addressing and CIDs \textit{IPFS} [online]. [cit. 2022-03-19]. Dostupné z: \url{https://docs.ipfs.io/concepts/content-addressing/}
             
    \bibitem{IPFSgateways}
    IPFS Gateway \textit{IPFS} [online]. [cit. 2022-03-19]. Dostupné z: \url{https://docs.ipfs.io/concepts/ipfs-gateway/}
    
    \bibitem{graphDatabasesIntroduction}
    Graph Databases for Beginners: Why Graph Technology Is the Future \textit{Neo4j} [online]. [cit. 2022-03-19]. Dostupné z: \url{https://neo4j.com/blog/why-graph-databases-are-the-future/}
     
    \bibitem{graphenTheorie}
    Graphentheorie \textit{Mathepedia} [online]. [cit. 2022-03-19]. Dostupné z: \url{https://mathepedia.de/Graphentheorie.html}
     
    
    \bibitem{graphAlgorithms}
    Graph Algorithms in Neo4j: Neo4j Graph Analytics \textit{Neo4j} [online]. [cit. 2022-03-19]. Dostupné z: \url{https://neo4j.com/blog/graph-algorithms-in-neo4j-neo4j-graph-analytics/}
     
        
    \bibitem{aboutNeo4j}
    The Fastest Path to Graph \textit{Neo4j} [online]. [cit. 2022-03-19]. Dostupné z: \url{https://neo4j.com}
            
    \bibitem{ACIDRules}
    Database Talk: What is ACID compliance? \textit{FairCom} [online]. [cit. 2022-03-19]. Dostupné z: \url{https://www.faircom.com/insights/database-talk-acid-compliance}
     
                
    \bibitem{CypherQL}
    Cypher Query Language \textit{Neo4j} [online]. [cit. 2022-03-19]. Dostupné z: \url{https://neo4j.com/developer/cypher/}
     
    
    
    
\end{thebibliography}




 

\end{document}