%%%%%%%%%%%%%%%%%%%%%%%%%
% Autor: Jakub Dokulil (kubadokulil99@gmail.com)
% Tato šablona byla vytvořena tak, aby pomocí ní mohli v systému LaTeX soutěžící sázet své práce a zároveň odpovídala požadavkům na formátování vyplývajícím z wordové šablony umístěné na webu soc.cz.
%
\documentclass[12pt, a4paper,
  %oneside,      %% -- odkomentujte, pokud chcete svou práci mít pouze jednostrannou, mezera pro hřbet pak automaticky bude pouze na levé straně
 twoside,        %% -- pro oboustranné práce, mezera pro hřbet následně střídá strany.
 openright
]{report}

%% Nutné balíčky a nastavení
%%%%%%%%%%%%%%%%%%%%%%%%%%%%

\newcommand\city{Pardubice} 
\newcommand\district{Pardubický}
\newcommand\specialization{Obor č. 18: Informatika}
\newcommand\school{SŠIE Delta}
\newcommand\consultant{RNDr. Koupil Jan, Ph.D.}
\newcommand\name{Krátký Kryštof}
\newcommand\publicationYear{2022}

\title{hiStories} %% -- Název tvé práce
\author{\name} %% -- tvé jméno
\date{\publicationYear} %% -- rok, kdy píšeš SOČku

\usepackage[top=2.5cm, bottom=2.5cm, left=3.5cm, right=1.5cm]{geometry} %% nastaví okraje, left -- vnitřní okraj, right -- vnější okraj

\usepackage[czech]{babel} %% balík babel pro sazbu v češtině
\usepackage[utf8]{inputenc} %% balíky pro kódování textu m
\usepackage[T1]{fontenc}
\usepackage{cmap} %% balíček zajišťující, že vytvořené PDF bude prohledávatelné a kopírovatelné

\usepackage{graphicx} %% balík pro vkládání obrázků

\usepackage{subcaption} %% balíček pro vkládání podobrázků

\usepackage{hyperref} %% balíček, který v PDF vytváří odkazy

\linespread{1.15} %% řádkování

\usepackage[pagestyles]{titlesec} %% balíček pro úpravu stylu kapitol a sekcí
\titleformat{\chapter}[block]{\scshape\bfseries\LARGE}{\thechapter}{10pt}{\vspace{0pt}}[\vspace{-22pt}]
\titleformat{\section}[block]{\scshape\bfseries\Large}{\thesection}{10pt}{\vspace{0pt}}
\titleformat{\subsection}[block]{\bfseries\large}{\thesubsection}{10pt}{\vspace{0pt}}

\setcounter{secnumdepth}{2}
\setcounter{tocdepth}{1}
\usepackage{fancyhdr}
\pagestyle{fancy}
\renewcommand{\headrulewidth}{1pt}

\usepackage{booktabs}

\usepackage{url}

%% Balíčky co se můžou hodit :) 
%%%%%%%%%%%%%%%%%%%%%%%%%%%%%%%

\usepackage{pdfpages} %% Balíček umožňující vkládat stránky z PDF souborů, 

\usepackage{upgreek} %% Balíček pro sazbu stojatých řeckých písmen, třeba u jednotky mikrometr. Například stojaté mí: \upmu, stojaté pí: \uppi

\usepackage{amsmath}    %% Balíčky amsmath a amsfonts 
\usepackage{amsfonts}   %% pro sazbu matematických symbolů
\usepackage{esint}     %% pro sazbu různých integrálů (např \oiint)
\usepackage{mathrsfs}

%% makra pro sazbu matematiky
\newcommand{\dif}{\mathrm{d}} %% makro pro sazbu diferenciálu, místo toho
%% abych musel psát '\mathrm{d}' mi stačí napsat '\dif' což je mnohem 
%% kratší a mohu si tak usnadnit práci

%% Bordel pro práci - můžeš smáznout :) 
%%%%%%%%%%%%%%%%%%%

\usepackage{lipsum} %% balíček který píše lipsum (nesmyslný text, který se používá pro kontrolu typografie)

%% Začátek dokumentu
%%%%%%%%%%%%%%%%%%%%


\begin{document}

\pagestyle{empty}
\pagenumbering{Roman}

\begin{titlepage}
    \bfseries{ %%% písmo na stránce je tučně
        \begin{center}
            \LARGE{STŘEDOŠKOLSKÁ ODBORNÁ ČINNOST} 
            \vspace{14pt}
            \large{\specialization} 

            \vspace{0.4 \textheight}

            \LARGE{ %%%%
                hiStories
            }%%%%

            \vspace{0.4\textheight}
        \end{center}
        
        \noindent\Large{\name}

        \noindent\Large{\district\hspace{\stretch{1}}  \city, \publicationYear} %% vyplň oficiální název kraje, město a rok
        
            
    } %%%
\end{titlepage}

\cleardoublepage%% Úvodní stránka s informacemi
{\bfseries %%% písmo na stránce je tučně
    \begin{center}
        \LARGE{STŘEDOŠKOLSKÁ ODBORNÁ ČINNOST}

        \vspace{14pt}
        {\large \specialization}

        \vspace{0.3 \textheight}

        \LARGE{ %%%%
        hiStories
        }

        \LARGE{ %%%%
        platforma pro sdílení historických fotek
        }%%%%

        \vspace{0.24\textheight}
    \end{center}  
}%%%
{\Large %%%
    \noindent\textbf{Jméno:} \name\\
    \textbf{Škola:} \school\\
    \textbf{Kraj:} \district\\
    \textbf{Konzultant:} \consultant\\
} %%%

\noindent \city, \publicationYear\cleardoublepage\noindent{\Large{\bfseries{Prohlášení}}}  %% uprav si koncovky podle toho na jaký rod se cítíš, vypadá to pak lépe :) 

\noindent Prohlašuji, že jsem svou práci SOČ vypracoval/a samostatně a použil/a jsem pouze prameny a literaturu uvedené v seznamu bibliografických záznamů.

\noindent Prohlašuji, že tištěná verze a elektronická verze soutěžní práce SOČ jsou shodné. 

\noindent Nemám závažný důvod proti zpřístupňování této práce v souladu se zákonem č. 121/2000 Sb., o právu autorském, o právech souvisejících s právem autorským a o změně některých zákonů (autorský zákon) ve znění pozdějších předpisů. 

\vspace{24 pt}

\noindent V Pardubicích dne 9.\ září 2022 \dotfill{}\hspace{\stretch{0.5}} 

\hspace{8cm} \name\cleardoublepage\vspace*{0.8\textheight}
\noindent{\Large{\bfseries{Poděkování}}}

\noindent
Chtěl bych poděkovat mému školiteli, prof. Farnsworthovi, za jeho úžasné tipy, triky a připomínky, bez kterých by nevznikla tato práce. Dále bych chtěl poděkovat mé rodině a přítelkyni, za to, že mě dostatečně zásobili kávou.

\cleardoublepage\noindent{\Large{\bfseries{Abstrakt}}}

\noindent Sem napíšeš svůj abstrakt. \lipsum[1] %% přepiš!!

\vspace{18pt}

\noindent{\Large{\bfseries{Klíčová slova}}}

\noindent Šablona, \LaTeX, SOČ, \dots 

\vspace{18pt}

\noindent{\Large{\bfseries{Abstract}}}

\noindent Write your abstract here! \lipsum[1] %% přepiš!!

\vspace{18pt}

\noindent{\Large{\bfseries{Keywords}}}

\noindent Template, \LaTeX, High school proffessional activity, \dots 

\cleardoublepage\tableofcontents

\pagenumbering{arabic}
\pagestyle{fancy}
\setcounter{page}{1}

\chapter*{Úvod}

\chapter{%Příloha A 
Spot diagramy a další }


\end{document}